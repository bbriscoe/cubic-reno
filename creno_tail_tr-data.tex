% !TeX root = creno_tr.tex
%=======================================================================================
%\section*{Acknowledgements}\label{Acks}
%The experiments in \autoref{fig:creno-v-reno-PIE} were conducted and plotted by Olga Albisser.

%=======================================================================================
\section{Conclusion}\label{Conclusion}

This report provides:
\begin{itemize}[nosep]
	\item a formula (\autoref{eqn:reno-friendly}) for the additive increase parameter of an AIMD algorithm as a function of its chosen multiplicative decrease factor that should maintain an equal flow rate with another AIMD flow, specifically a Reno flow. 
	
	\item a derivation of the formula that relies on fewer assumptions and is more rigorous than that in Floyd \emph{et al}~\cite{Floyd00:Eqn_v_AIMD_cc}. It applies to tail drop buffers whereas that in Floyd \emph{et al} relied on an AQM with deterministic marking. Nonetheless the formula turns out to be the same.
	
	\item a testbed validation of the formula (at least for the case where the MD factor is 0.7) over a range of 25 different path characteristics (5 rates and 5 base RTTs) with a tail-drop bottleneck buffer.
	
	\item a testbed evaluation of the formula over the same range of scenarios but with a PIE AQM at the bottleneck. This shows that the AI factor works correctly over a PIE AQM, even though it was derived assuming tail-drop.
\end{itemize}

These result show that, with the widely deployed decrease factor of \(b_c=0.7\), the Reno-Friendly mode in CUBIC is sufficiently well modelled by \autoref{eqn:reno-friendly} that the Additive Increase factor it produces (\(a_c = 0.53\)) ensures that TCP CUBIC competes roughly equally with Reno across its intended operating range, whether with a tail-drop queue or a single-queue AQM (PIE) at the bottleneck.